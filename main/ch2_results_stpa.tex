\cleardoublepage
\chapter{Optimized Diffractive Optical Elements by STPA}
In this chapter we will see some examples of tables and figures.
\section{Process of optimization}
\subsection{Figure of Merit}
\subsection{Design parameters}
\subsection{Initial design}
\subsection{Gradient calculation}
\section{One-dimensional diffractive optical elements}
\subsection{Reconstruction results}
\subsection{Experimental results}
\section{Limits of optimization by STPA}
Let's see how to make a well designed table.

\begin{table}[tb]
\caption[A floating table]{A floating table.}
\label{tab:esempio}
\centering
\begin{tabular}{ccc}
\toprule
name & weight & food \\ 
\midrule
mouse	& 10 g	& cheese \\
cat	& 1 kg	& mice \\
dog	& 10 kg	& cats \\
t-rex	& 10 Mg	& dogs \\
\bottomrule 
\end{tabular}
\end{table}
\textit{}

The table~\ref{tab:esempio} is a floating table and was obtained with the following code:
\begin{lstlisting}
\begin{table}[tb]
\caption[A floating table]{A floating table.}
\label{tab:example}
\centering
\begin{tabular}{ccc}
\toprule
	name 	& weight & food	  \\ 
\midrule
	mouse	& 10  g	 & cheese \\
	cat		&  1 kg	 & mice	  \\
	dog		& 10 kg	 & cats   \\
	t-rex	& 10 Mg	 & dogs	  \\
\bottomrule 
\end{tabular}
\end{table}
\end{lstlisting}

\lipsum[1-2]
\lipsum[3-8]
